\title{CSC 458 Assignment 2}
\author{Nick Graham}
\date{\today}

\documentclass[12pt]{article}

\usepackage{graphicx}

\begin{document}
\maketitle

\section*{Introduction}
Assignment Two had us look at different implementations of locks and compare them
to see which ones performed best at different numbers of threads under different
loads. The locks we tested were the built C++11 mutex, a Test and Set lock, a Test
and Set lock with backoff, a Test and Test and Set lock, a Ticket lock, a Ticket
lock with backoff, a MCS lock, a K42 MCS lock, a CLH Lock, and the K42 CLH lock.
A simple counter program was used again to test the locks, having multiple threads
access the same counter. The program was tested on both the x86 machines as well as
the IBM machines.

\section*{Data}

\subsection*{x86}
These plots show the number of increments per second through the heatmap. In these
plots the larger the number (yellow in color) the better the lock performed

\includegraphics[scale=.33]{{"Graphs/100 Count per Thread"}.png}

\includegraphics[scale=.33]{{"Graphs/1000 Count per Thread"}.png}

\includegraphics[scale=.33]{{"Graphs/10000 Count per Thread"}.png}

\includegraphics[scale=.33]{{"Graphs/100000 Count per Thread"}.png}

\subsection*{IBM}
These plots show the number of increments per second through the heatmap. In these
plots the larger the number (yellow in color) the better the lock performed The
100000 increment test was not run on the IBM machine due to the large amount of time
it took to run, and I did not want to monopolise the machine for any longer then I
already had.

\includegraphics[scale=.33]{{"Graphs/100 Count per Thread IBM"}.png}

\includegraphics[scale=.33]{{"Graphs/1000 Count per Thread IBM"}.png}

\includegraphics[scale=.33]{{"Graphs/10000 Count per Thread IBM"}.png}

\section{Analysis}
All methods returned the expected counter values of $numberOfThreads \times IncrementNumber$
which is as expected. This means that all of the locks are functioning as designed and
only allowing one thread to hold the lock at a given time.

Overall on the x86 machine the Test and Set with Backoff had the best performace
overall. At low increment numbers and lower thread numbers (between 4 and 24 threads)
the C++11 Mutex, TAS, and TATAS did outperform the Test and Set with backoff, however
not but much, and at large Increment counts, the Test and Set with Backoff was the best
for any number of threads greater then 1, by a very large margin.

On the IBM Machine the performance overall was worse. For lower increment counts
the C++11 Mutex, and all of the TAS locks were about equivelent, however much like
on IBM as the increment count increased the Test and Set with Backoff pulled ahead
as the best performing lock.

I believe that the IBM has worse performace due to its implementation on the atomic
instructions. I believe in class it was discussed that a lot of the atomic instructions
on IBM are implemented as Load Linked/Store Conditional which has a high chance of
failing when there is a large amount of contention on the object.

\subsection{Backoff Tuning}
I do not think that my backoff tuning is the best it could be. For the Test And Set
Lock I found that a starting point of 1 ms, with a max of 1000ms and a multiplier of
2 gave the best performance. For the Ticket Lock a base of 1 ms was the best I could do.
In both cases the base was the lowest I could go for my sleeping function. I believe that
if I had instead slept for shorter amounts of time, the locks would have been able to
perform much better.

\section{Conclusions}\label{conclusions}
Overall my tests showed that the Test and Set with backoff was the most efficient algorithm
for locking. I also found the IBM machine had worse performance overall. There are
a few sources of error in my implemention mainly in my use of atomics and my sleep
function. I do not always use the lowest possible memory order in the locks, which will
increase the time it takes to access and release the lock, hurting the overall performance,
and my sleep function for the locks with backoff had a minimum amount of time that
I believe was too high, which made them not perform as optimally as they could have. 
\end{document}
